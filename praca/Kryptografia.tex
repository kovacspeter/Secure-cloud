\chapter{Základné pojmy a označenia}

Prvá kapitola slúži ako teoretický úvod do práce. Zadefinujeme základné pojmy kryptografie, vysvetlíme, čo je symetrické a asymetrické šifrovanie a ako fungujú. Zároveň si vysvetlíme, čo sú hašovacie funkcie a cloudové úložiská. Budeme vychádzať z knihy Handbook of Applied Cryptography \cite{HOAC}.

\section{Kryptografia}
	Základným cieľom kryptografie je ukladanie a prenášanie dát vo forme, ktorú dokáže spracovať iba tá entita, ktorej sú dáta určené.

\section{Množiny a operácie}
	Zadefinujeme si základné množiny a operácie nad nimi, ktoré budeme používať.
	\begin{itemize}
	\item Množinu $\mathcal A$ budeme nazývať abecedou. Abecedou je napríklad slovenská abeceda alebo $\mathcal A$ =  $\{0,1\}$ je binárnou abecedou.
	\item Množina $\mathcal M$ je množina všetkých možných správ nad danou abecedou $\mathcal A$. Napríklad nad abecedou $\mathcal A = \{0,1\}$ pri správach maximálnej dĺžky 2 je  $ \mathcal M = \{00,01,10,11\}$.
	\item Množina $\mathcal C$ obsahuje všetky šifrované správy nad danou abecedou $\mathcal A$.  
	\item Množinu $\mathcal K$ nazveme množina kľúčov. Prvok $k \in\mathcal K$ nazveme kľúč.
	\item Každý prvok $e \in \mathcal K$ jednoznačne určuje bijekciu z $\mathcal M$ do $\mathcal C$. Túto transformáciu budeme značiť $E_e$ a budeme ju nazývať šifrovacou funkciou.
	\item Nech $D_d$ je bijektívna transformácia z $\mathcal C$ do $\mathcal M$ pomocou prvku  $d \in \mathcal K$, potom  $D_d$ nazveme dešifrovacou funkciou.
	\item Keď aplikujeme transformáciu $E_e$ na správu $m \in \mathcal M$ budeme hovoriť, že šifrujeme správu $m$. Ak aplikujeme $D_d$ na $c \in \mathcal C$ budeme hovoriť o dešifrovaní.
	\item Šifra alebo aj šifrovacia schéma sa skladá z množiny $\{E_e : e \in K\}$ a množiny $\{D_d : d \in K\}$, kde platí, že pre kazdé $e \in K$ existuje $d \in K$ také, že $D_d = E_e^{-1}$ a teda platí aj $D_d(E_e(m)) = m$ pre všetky $m \in \mathcal M$.  
	\end{itemize}

\newtheorem{mydef}{Definícia}
\section{Symetrické šifrovanie}
\begin{mydef}
	Nech šifrovacia schéma pozostáva z množín $\{E_e : e \in\mathcal K\}$ a $\{D_d: d \in\mathcal K\}$, kde $\mathcal K$ je množina všetkých kľúčov. Takúto schému nazveme symetrickou, ak pre každý pár $(e,d)$ platí, že je "ľahké", vypočítať $d$ pomocou $e$, a opačne. 
\end{mydef}
	Najčastejšie používame $e = d$. Symetrické šifry sú zväčša veľmi rýchle, takže dokážu zašifrovať veľa dát za krátky čas a kľúče sú relatívne krátke. Ďaľšou výhodou je možnosť zapúzdrenia, teda na jednu správu môže byť použitých viac šifier, vďaka čomu môžu dosahovať vačšiu mieru bezpečnosti. Pri komunikácii dvoch entít býva dobrým zvykom meniť kľúče relatívne často.

\section{Asymetrické šifrovanie}
\begin{mydef}
	Nech $\{E_e : e \in\mathcal K\}$ je množina šifrovacích funkcií a nech  $\{D_d : d \in\mathcal K\}$ je množina príslušných dešifrovacích funkcií a $\mathcal K$ je množina všetkých kľúčov. Nech pre každý pár $(E_e,D_d)$ platí, že je "výpočtovo nemožné"\footnote{Pojem "výpočtovo nemožné" je postavený na predpoklade, že problém, ktorý riešime nie je efektívne riešiteľný, tzn. žiadny efektívny algoritmus nevie nájsť s nezanedbateľnou pravdepodobnosťou riešenie v potrebnom čase.} získať správu $m \in\mathcal M$ pomocou $c \in\mathcal C$ a $E_e$, keď platí $E_e(m) = c$. Takéto šifrovanie nazveme asymetrické.
\end{mydef}
	Z definície zároveň vyplýva, že keď máme $e \in\mathcal K$, tak je nemožné získat príslušný kľúč $d$ taký, aby platilo $D_d(E_e(m)) = m$. Aby bola táto vlastnosť zabezpečená, kryptografické funkcie sú založené na matematických problémoch akými sú napr. faktorizácia alebo výpočet diskrétneho logaritmu. Kľúč $d$ budeme nazývať privátnym a $e$ verejným kľúčom. 
	
	Pri využití asymetrickej kryptografie nám stačí uchovávať bezpečne len privátny kľúč a kľúče netreba meniť tak často ako pri symetrických šifrách. Nevýhodou oproti symetrickému šifrovaniu môže byť väčšia veľkosť kľúčov a nižšia rýchlosť šifrovania.
		

\section{Hašovacie funkcie}
\begin{mydef}
	Hašovacou funkciou $\mathcal H : \mathcal I \rightarrow \mathcal O$ nazveme funkciu zobrazujúcu z množiny vstupov ľubovoľnej dĺžky $\mathcal I$ na množinu reťazcov bitov pevnej dĺžky $\mathcal O$. Prvky množiny $\mathcal O$ budeme nazývať haše. 
		
\end{mydef}
	Vačšina kryptografických schém využívajúcich hašovacie funkcie využíva takzvané kryptografické hašovacie funkcie. Kryptografická hašovacia funkcia obvykle spĺňa nasledujúce vlastnosti:
	\begin{itemize}
		\item Jednosmernosť: k danému hašu $o \in\mathcal O$ je "výpočtovo nemožné"\ získať akýkoľvek vstup $i \in\mathcal I$ spĺňajúci $\mathcal H(i) = o$.
		\item Odolnosť voči kolíziám: je "výpočtovo nemožné"\ nájsť dva rôzne vstupy s rovnakým hašom $i_1 \ne i_2,$ kde $i_1,i_2 \in\mathcal I$ pre ktoré platí, že $\mathcal H(i_1) = \mathcal H(i_2)$.
	\end{itemize}
	Kryptografické využitie nachádza v digitálnych podpisoch, konštrukciách autentizačných kódov, pri ukladaní hesiel alebo pri kontrole integrity údajov.


\section{Použité šifry}

	V tejto časti predstavíme šifry, ktoré budeme používať v našom riešení. 
	
	\subsubsection{AES}
		

		Veľmi rýchla bloková šifra založená na permutáciách a substitúciách, s veľkosťou bloku 128bitov. Dĺžka kľúča sa rôzni od 128 bitov cez 192 bitov až po 256 bitov. V našom riešení ju budeme používať v jej 128-bitovej forme na účely symetrickej kryptografie.
%Algoritmus šifrovania je nasledovný: 
%		\begin{enumerate}
%			\item Expanzia kľúča - podkľúče sú odvodené z kľúču
%			\item Inicializácia - xorujeme stav s podkľúčom
%			\item Cykly - prebiehajú zámeny bajtov, prehadzovanie riadkov, kombinovanie stĺpcov a xorovanie stavu s podkľúčom 
%			\item Posledný cyklus - zámena bajtov, poprehadzovanie riadkov a xorovanie stavu s podkľúčom 
%		\end{enumerate}

		Podľa amerického úradu pre štandardizáciu(NIST), by šifra AES s dĺžkou kľúča 128 bitov mala byť bezpečná aspoň do roku 2030. Jej implementáciu je možné nájsť napríklad v knižnici SJCL \cite{SJCLgit}, ktorú budeme používať. Viac informácií o nej môžeme nájsť v štandarde FIPS - 197 \cite{FIPS197}.
	\subsubsection{ElGamalova šifrovacia  schéma}
		Algoritmus založený na probléme diskrétneho logaritmu slúžiaci na asymetrické šifrovanie. Plánujeme ho využívať pri zdieľaní súborov, aj pri šifrovaní hesiel k súborom. Budeme využívať krivku P-256, ktorá je štandardizovaná organizáciou NIST a jej predpokladaná bezpečnosť je aspoň do roku 2030. Jeho implementáciu tiež môžeme nájsť v SJCL \cite{SJCLgit}.
		
\section{Cloudové úložiská}
	Cloudové úložisko je služba, ktorá nám umožňuje manipulovať s diskovým priestorom prenajatým od poskytovateľa služby. Táto služba by mala byť škálovateľná, čo znamená, že vieme jednoducho zväčšiť priestor, za ktorý platíme. Ďalej by mala zabezpečovať prístupnosť dát, čo zahŕňa ochranu voči strate a poškodeniu dát.
	
	