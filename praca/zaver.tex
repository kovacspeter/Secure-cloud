\chapter*{Záver}
\markboth{Záver}{}
\addcontentsline{toc}{chapter}{Záver}

%V závere je potrebné v stručnosti zhrnúť dosiahnuté výsledky vo vzťahu k stanoveným 
%cieľom. Rozsah záveru je minimálne dve strany. Záver ako kapitola sa nečísluje. 

Práca sa zaoberala implementáciou webovského rozhrania, ktoré poskytuje šifrovanie dát v cloude. 
V úvode sme sa venovali kryptografii, popísali sme jej základné pojmy a typy šifrovania. Ďalej sme predstavili 
zoznam existujúcich riešení pre bezpečné zdieľanie dát. Prvú skupinu riešení tvoria služby využívajúce vlastné cloudové riešenia, 
napr. úložisko Mega, do druhej skupiny sme zaradili služby využívajúce už dostupné cloudy (Google Drive, DropBox a iné).
Nevýhodou prvej skupiny je nutnosť prejsť na neznáme cloudové prostredie a nevýhodou druhej je potreba inštalácie dodatočného softvéra. 

My sme zvolili riešenie, ktoré tieto dva nedostatky eliminuje a kombinuje výhody oboch prístupov. Vytvorili sme službu SecureCloud, 
ktorá pre použitie vyžaduje iba
moderný prehliadač, internetové pripojenie a účet populárneho cloudu Google Drive. Pre integráciu s inými cloudovými úložiskami 
by stačilo implementovať časti, ktoré pracujú s novým cloudom a napojiť ich na naše webové rozhranie. 

V tretej kapitole sme navrhli spôsob fungovania SecureCloudu. Popísali sme jeho funkcionalitu, 
ktorej prednosťou je, že nevyžaduje zdieľanie informácií v otvorenom tvare so serverom ani cloudom. V službe Mega musí používateľ sprostredkovať 
svojmu adresátovi heslo na odšifrovanie súborov, kým v našom riešení je potrebné len zadať e-mail adresáta a zvoliť súbor na zdieľanie. To umožňuje
vyhnúť sa distribúcii hesla po potenciálne nezabezpečených kanáloch.

V poslednej kapitole sme podrobne popísali implementáciu nášho webovského rozhrania, uviedli sme použité technológie, 
časti zdrojového kódu SecureCloudu a vysvetlili sme ich fungovanie.

Na prácu je možné nadviazať implementáciou ďalších funkcionalít, ako je napríklad šifrovaná kolaborácia v reálnom
čase alebo podpora pre zdieľané tajomstvo. Ďalším dôležitým aspektom je používateľská prívetivosť, na ktorú sme 
sa v tejto práci nesústredili. 