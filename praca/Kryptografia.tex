\chapter{Kryptografia}

Nasledujúca kapitola slúži ako teoretický úvod do kryptografie. Vymenujeme jej ciele, zadefinujeme základné pojmy a vysvetlíme, čo je symetrické a asymetrické šifrovanie a ako fungujú. Budeme vychádzať z knihy Handbook of Applied Cryptography \cite{HOAC}.

\section{Kryptografia}
	Kryptografia sa zaoberá metódami ukladania a prenášania dát vo forme, ktorú dokáže spracovať iba taká entita, ktorej sú dáta určené.
	\subsubsection{Ciele kryptografie}
	Na úvod popíšeme základné kryptografické ciele ako dôvernosť, integritu a autentickosť. Z cieľov vynecháme dostupnosť, ktorú kryptografia ako taká nedokáže zabezpečiť.
		\begin{itemize}
		\item Dôvernosť je vlastnosť, ktorá nám zaručuje, že k dátam sa dostanú len také entity, ktorým bola správa určená a nikto iný. Dôvernosť dát budeme zabezpečovať šifrovaním. 
		\item Integrita je vlastnosť, ktorá hovorí o modifikácii dát. Aby sme zaručili integritu dát, musí byť zaručená možnosť detegovať manipuláciu s dátami neoprávnenými entitami.
		\item Autentickosť je vlastnosť, ktorá hovorí o pôvode entity alebo dát. Teda keď Anička pošle správu Bohušovi, Bohuš bude môcť overiť, že správa je skutočne od Aničky. 
		\end{itemize}

\section{Množiny a operácie}
	Zadefinujeme si základné množiny a operácie nad nimi, ktoré budeme používať.
	\begin{itemize}
	\item Množinu $\mathcal A$ budeme nazývať abecedou. Abecedou je napríklad slovenská abeceda alebo $\mathcal A$ =  $\{0,1\}$ je binárnou abecedou.
	\item Množina $\mathcal M$ je množina všetkých možných správ nad danou abecedou $\mathcal A$. Napríklad nad abecedou $\mathcal A = \{0,1\}$ pri správach maximálnej dĺžky 2 je  $ \mathcal M = \{00,01,10,11\}$.
	\item Množina $\mathcal C$ obsahuje všetky šifrované správy nad danou abecedou $\mathcal A$.  
	\item Množinu $\mathcal K$ nazveme množina kľúčov. Prvok $k \in\mathcal K$ nazveme kľúč.
	\item Každý prvok $e \in \mathcal K$ jednoznačne určuje bijekciu z $\mathcal M$ do $\mathcal C$. Túto transformáciu budeme značiť $E_e$ a budeme ju nazývať šifrovacou funkciou.
	\item Nech $D_d$ je bijektívna transformácia z $\mathcal C$ do $\mathcal M$ pomocou prvku  $d \in \mathcal K$, potom  $D_d$ nazveme dešifrovacou funkciou.
	\item Keď aplikujeme transformáciu $E_e$ na správu $m \in \mathcal M$ budeme hovoriť, že šifrujeme správu $m$. Pokiaľ aplikujeme $D_d$ na $c \in \mathcal C$ budeme hovoriť o dešifrovaní.
	\item Šifra alebo aj šifrovacia schéma sa skladá z množiny $\{E_e : e \in K\}$ a množiny $\{D_d : d \in K\}$, kde platí, že pre kazdé $e \in K$ existuje $d \in K$ také, že $E_e = D_d$ a teda platí aj $D_d(E_e(m)) = m$ pre všetky $m \in \mathcal M$.  
	\end{itemize}

		
\section{Symetrické šifrovanie}
	Nech šifrovacia schéma pozostáva z množín $\{E_e : e \in K\}$ a $\{D_d: d \in K\}$ kde $\mathcal K$ je množina všetkých kľúčov. Takúto schému nazveme symetrickou pokiaľ pre každý pár $(e,d)$ platí, že je "ľahké", vypočítať $d$ pomocou $e$, a opačne. Najčastejšie používame $e = d$. Symetrické šifry sú zväčša veľmi rýchle, takže dokážu zašifrovať veľa dát za krátky čas a taktiež kľúče sú relatívne krátke. Symetrické šifry môžu byť zapúzdrené, teda na jednu správu môže byť použitých viac šifier, vďaka čomu môžu dosahovať vačšiu mieru bezpečnosti. Na druhú stranu pri komunikácii dvoch entít býva dobrým zvykom meniť kľúče relatívne často a taktiež kľúč musí ostať v bezpečí počas celej komunikácie.

\section{Asymetrické šifrovanie}
	Nech $\{E_e : e \in K\}$ je množina šifrovacích funkcií a nech  $\{D_d : d \in K\}$ je množina príslušných dešifrovacích funkcií a $\mathcal K$ je množina všetkých kľúčov. Nech pre každý pár $(E_e,D_d)$ platí, že je výpočtovo "nemožné" získať správu $m \in\mathcal M$ pomocou $c \in\	 C$ a $E_e$, keď platí $E_e(m) = c$. 
	\\Definícia nám hovorí, že keď máme $e \in\mathcal K$ tak je nemožné získat príslušný kľúč $d$ taký aby platilo $D_d(E_e(m)) = m$. Aby sme zabezpečili túto vlastnosť vačšina šifrovacích funkcií je založená na matematických problémoch ako je faktorizácia alebo výpočet diskrétneho logaritmu. Kľúč $d$ budeme nazývať privátnym a $e$ verejným kľúčom. Pri využití asymetrickej kryptografie nám stačí uchovávať privátny kľúč a taktiež kľúče netreba meniť tak často ako pri symetrických šifrách. Nevýhodou oproti symetrickému šifrovaniu môže byť velkosť kľúčov a rýchlosť šifrovania ktorá býva často nižšia.	
	
\section{Vyuzijeme}
Tu by sa hodilo popisat nejake šifrovacie veci ktoré budem pouzívať. Napríklad ECC, AES poprípade nejake password derivátory scrypt/bcrypt/pbkdf2 alebo crypto-hash funkcie.
		