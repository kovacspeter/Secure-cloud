\chapter*{Úvod}
\markboth{Úvod}{}
\addcontentsline{toc}{chapter}{Úvod}

V práci sa budeme venovať vývoju webového rozhrania pre bezpečné zdieľanie dokumentov  v cloude.
Ukážeme si, ako je možné využiť existujúce cloudové riešenia a zostrojiť vrstvu medzi nimi a používateľom, ktorá
bude zabezpečovať kryptografiu a bezpečnosť dát. Systém sa pokúsime navrhnúť tak, aby server a cloud mali čo najmenej 
 informácií o súboroch alebo o kľúčoch v otvorenom tvare.
Nakoniec budeme schopní nahrávať, sťahovať a zdieľať šifrované dáta uložené v Google Drive cloude.

Systém s takýmito vlastnosťami môže mať široké uplatnenie. Firemní zamestnanci často používajú cloudové úložiská na zdieľanie dát, ktoré 
môžu obsahovať obchodné tajomstvá či iné citlivé informácie. V takom prípade je dôležité zabezpečiť dôvernosť informácií, čo je hlavnou úlohou nášho riešenia.
Téma bezpečného zdieľania dát sa dostala do povedomia aj laickej spoločnosti v roku 2013 po zverejnení dokumentov o sledovacích aktivitách americkej NSA a znova v roku 2014 po úniku intímnych fotografií z iCloudu niektorých hollywoodskych hviezd. 
Vďaka tomu môže byť naše riešenie zaujímavé nielen v korporátnej sfére ale aj pre bežného používateľa, ktorý chce svoje dáta chrániť.

Prácu rozdelíme do štyroch kapitol. V prvej zadefinujeme základné pojmy, vysvetlíme, čo je šifrovanie
a aké šifrovanie budeme používať. V ďalšej kapitole
poskytneme prehľad existujúcich riešení pre bezpečné zdieľanie dát, popíšeme ich funkcionalitu a vymenujeme ich nevýhody.
Ďalej ukážeme, ako bude vyzerať naše riešenie a v čom sa bude odlišovať od ostatných.

V tretej kapitole navrhneme spôsob jeho fungovania.  V úvode návrhu popíšeme, čo musíme spraviť predtým, ako budeme schopní pracovať s dátami. 
Ďalej si ukážeme, ako bude fungovať nahrávanie, sťahovanie súborov a práca so zdieľanými dátami.

Posledná kapitola sa bude zaoberať implementáciou návrhu. Uvedieme v nej technológie,
ktoré sme využili na strane klienta, a ktoré sme použili na strane servera. Zároveň ukážeme najdôležitejšie segmenty zdrojového kódu a vysvetlíme ich fungovanie. 

%Úvod je prvou komplexnou informáciou o práci, jej cieli, obsahu a štruktúre. Úvod sa 
%vzťahuje na spracovanú tému konkrétne, obsahuje stručný a výstižný opis 
%problematiky, charakterizuje stav poznania alebo praxe v oblasti, ktorá je predmetom 
%školského diela a oboznamuje s významom, cieľmi a zámermi školského diela. Autor 
%v úvode zdôrazňuje, prečo je práca dôležitá a prečo sa rozhodol spracovať danú tému. 
%Úvod ako názov kapitoly sa nečísluje a jeho rozsah je spravidla 1 až 2 strany.
